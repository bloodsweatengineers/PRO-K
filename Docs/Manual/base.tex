\documentclass{article}

\usepackage{siunitx}
\usepackage{listings}

\title{User Manual technical documentation}

\author{Folkert Kevelam \and Kerim Kilic \and Dennis van den Berg}

\begin{document}
\pagenumbering{gobble}
\maketitle
\pagenumbering{roman}
\newpage
\tableofcontents
\newpage

\pagenumbering{arabic}

\section{Introduction}
This manual servers as an introduction and technical reference of the communication
API of the PWM controller software. The PWM controller is reachable via 2 different
communication methods. One being terminal communication with a UART/USART connection
and the other being the graphical user interface, where the actual commands are hidden
behind a more user-friendly interface. This manual is therefore split into 2 different
parts.

\section{Terminal communication}
Terminal communication can be achieved using a VCOM port and a working USB controller.
The user sends commands through the VCOM port which is then intrpreted by the kernel
of system the user is using, and send as USB commands to the Arduino Nano. The Arduino Nano
intreprets the USB commands and translates them to UART commands.

The syntax for sending a command is as follows: \\
\begin{lstlisting}
{command}_{channel} {value}\r\n
\end{lstlisting}

Where command is a valid command listed in section \ref{sec:Commands}. The channel
may be an optional argument as it isn't always necessary. The value argument
decides the kind of command it is, if no value is send by the user the firmware
will interpret it as a request for information and return the value of the specific
parameter and specific channel if applicable. The value send needs to be between
certain ranges, those ranges can also be found in the command section \ref{sec:Commands}

Finally, the firmware always sends a return code. This is always an `OK` or `REJ`.
All messages are terminated using the carriage return and newline characters.
The `REJ` return code may also add a numeric error code, which can be useful
for finding errors in send commands.

\section{Graphical User Interface}

\section{Commands}
\label{sec:Commands}
This sections holds all the different commands and their binary representation.
It also holds the range in which value are accepted for the given command, and whether
users may use specific channels.\newline
\begin{tabular}{c c c c}
	\hline
	Command & Binary Representation & Has channels & value range\\
	\hline \hline \\
	frequency & 0x01 & No & (0.01 - 80)/400\si{\hertz}\\
	amplitude & 0x1X & Yes & 0 - 100 \si{\percent}\\
	phaseshift & 0x3X & Yes & 0 - 360 \si{\degree}\\
	pwm\_frequency & 0x20 & No & 1,4,16,64,256,1024 \\
	start & 0x04 & No & N/A \\
	stop & 0x05 & No & N/A \\
	info & 0x00 & No & N/A \\
	ping & 0xFF & No & N/A \\
	prepare & 0x02 & No & N/A \\
	execute & 0x03 & No & N/A \\
	gather & 0x40 & Yes & 0-256 samples \\
	vfd & 0x06 & No & N/A \\
	enable & 0x50 & Yes & true or false \\
	\hline
\end{tabular}

\end{document}

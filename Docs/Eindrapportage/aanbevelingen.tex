\section{Aanbevelingen voor de opvolger}
De aanbevelingen voor de opvolger bestaan uit procesmatige, hardwarematige en softwarematige aanbevelingen. De belangrijkste procesmatige aanbeveling voor de opvolger is dat het belangrijk is om in de eerste weken van het project alle deliverables en eisen vast te stellen en een moving target te voorkomen. De hardwarematige aanbevelingen gaan over de bestaande hardware en de softwarematige aanbevelingen gaan over de python code.
\subsubsection{hardware}
Er zitten een aantal fundamentele bugs in de bestaande hardware. Om te beginnen is bij de buckregulator de output en de feedback pin omgewisseld. De buckregulator werkt wel maar is minder betrouwbaar door deze fout. De 5V lineaire voeding op de PCB staat parallel aan de 5V lineaire voeding op de Arduino Nano. In de huidige toestand kan de PC niet communiceren met de Arduino zonder externe bedrading toe te voegen. Verder mist er een Aref condensator op de PCB, hierdoor werkt, zonder externe bedrading en een condensator de ADC niet. De pinout van de Arduino Nano die op de PCB moet klopt niet, de PWM pinnen op de Arduino komen niet overeen met de stuurpinnen op de PCB. Ook een belangrijke aanbeveling is het gebruik maken van krachtigere hardware die in dezelfde prijscategorie valt als een Arduino Nano, zoals een microcontroller van STM of NXP.
\subsubsection{software}
Voor de opvolger van dit project wordt er aangeraden om de structuur van de software te analyseren. Door de modulariteit van de software is het eenvoudig om uitbreidingen toe te voegen. De software van het user interface kan uitgebreid worden met meer features, zoals realtime plots van de spanning en stroom en een FFT om de verschillende harmonische in het signaal te bepalen.

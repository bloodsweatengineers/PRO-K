\section{Afwijking van oorspronkelijke planning en resultaten}
\subsubsection{Verwachting}
Het project is anders gelopen dan de initiële planning en resultaten. De projectomschrijving \footnote{\href{https://blackboard.hhs.nl/webapps/blackboard/execute/content/file?cmd=view&content_id=_2697384_1&course_id=_73453_1&framesetWrapped=true}{Projectomschrijving H-pwm\_controller\_1.pdf}, Blackboard} wekte in de eerste instantie de indruk dat er naast de software die is ontwikkeld ook Hardware ontwikkeld moest worden. Na nadere vergaderingen met de opdrachtgever kwam naar boven dat enkel de software nog ontwikkeld moet worden. 

\subsubsection{Planning}
Het project is anders verlopen dan initiële gepland was. In de initiële planning was er opgenomen om in blok 3 een gros van zowel de embedded code als de UI code af te ronden. Echter, omdat er gedurende de eerste 7 weken van het project verschillende partijen bij het project werden betrokken is er veel onduidelijkheid en discussie ontstaan tussen de klant, de technisch adviseur en de engineers. Pas in het einde van blok 3 is er een compromis gesloten tussen de klant (Ir. P. van Duijssen) en de technisch adviseur (Ir. F. Theinert) van de software en het protocol wat opgeleverd moet worden. Dit heeft voor vertraging gezorgd bij de engineers in de ontwikkeling van de software en het communicatieprotocol. Door deze vertraging is er bijvoorbeeld minder aandacht besteedt aan het onderzoek naar daadwerkelijk alle fouten in de hardware.

\subsubsection{Resultaten}
Er zijn ook een aantal afwijkingen in de eindresultaten ten opzichte van het plan van aanpak. Een van deze aanpassingen is dat de amplitude niet bij alle vier de kanalen individueel instelbaar is. Dit heeft te maken de specificaties van de microcontroller waar de projectgroep aan gebonden was. Deze microcontroller had te weinig CPU kracht om deze dynamische berekening te maken en te weinig RAM om een statische berekening te maken. De microcontroller zou 1 kB meer RAM minimaal nodig moeten hebben om deze dynamische berekening te kunnen maken. Uiteindelijk is ervoor gekozen om i.v.m. de specificaties, drie kanalen dezelfde amplitude te geven en bij het vierde kanaal een afzonderlijke amplitude in te stellen. Ook is er besloten om de PWM frequentie niet meer instelbaar te maken. Dit is gedaan, omdat een zo hoog mogelijke PWM frequentie gewenst is voor de beste resolutie aan de uitgang van de Universal 4 Leg. Naast resultaten die uit het einddesign zijn gehaald, zijn er ook bepaalde dingen toegevoegd die niet gespecificeerd waren in het plan van aanpak. Zo is er een zogenaamde Variable Frequency Drive functie geïmplementeerd die langzaam de amplitude en frequentie verhoogt. Dit kan handig gebruikt worden bij het aansturen van synchrone motoren.

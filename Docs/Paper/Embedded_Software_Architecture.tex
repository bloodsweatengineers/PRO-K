\section{Embedded software architecture}
A large part of the project pertains the control and generation of PWM waveforms described by
user input. To that end, the embedded software must be able to perform its generation duties whilst
recieving or outputting paramters/data. There are multiple ways of achieving said multitasking such as
polling or interrupt-driven execution. Since the required hardware does not have the ability to prioritise
specific interrupts, a pure interrupt-driven approach was ruled out, instead, it was chosen to focus on a dual apporach. The routines with a deadline such as the waveform generation would work with interrupts while the routines without a deadline such as user interaction would work with a polling system. The doxygen documentation for the Embedded Software can be found in the Technical Appendices.

\subsection{Polling state}
The interactivity between the user and device is implemented using serial communication.
This means that beside controlling the board itself, the microcontroller needs to reserve
time to get user input. As described above, the general approach used in this project is
the hybrid interrupt-driven/polling approach. The waveform is generated and output using
interrupts, whilst the user interaction is done using polling.

Polling consists of checking if a change has been made every once in a while. This
is usually done in the main loop of the program. Every time the system loops a routine
checks a value again and if nothing has changed, it continues. 

There are 2 ways to poll changes without interrupts, using blocking or non-blocking routines.
When a polling routine is the only routine that needs to poll, blocking routines are
very easy to use and reliable, however, as soon as multiple routines need to poll, a single
routine can't block the microcontroller. This means that more complex projects need to have
non-blocking routines. A computer implements a sort of hybrid between blocking and non-blocking.
It is possible to get a blocking program on a mainstream personal use computer, however, the kernel
just switches between programs(processes) when a program is blocking.

\subsubsection{Implementation}
The implementation of non-blocking polling routines in embedded firmware, differs a little
from the implementation of its blocking routine brother. The base principle is to act with
a part of the data/change, instead of waiting for the whole data package. An example being
the user interaction. In a blocking routine the parser would either read or parse the entire
command in one go, effectively waiting every time no data is present in the buffer.
A non-blocking routine would read or parse per data character, and would remember an internal
state of the parser so as to remember its progress.

The implementation of the non-blocking polling routines uses a single read/parse whilst
holding the state of the routine.

\subsection{Input and output formatting}
Since the program needs to work in binary as well as string mode, the software needs to
be able to switch between the 2. This is done by switching a button on the board.

\subsection{Waveform generation}
The waveform generation for the program has several stages. These stages are used
to relieve the CPU, and provide methods for changing parameters during runtime.
\subsubsection{Stage 1: Setting parameters}
The parameters are set to static variables that can be accessed by the Main timer interrupts
and the PWM timer interrupts. The amplitude-changed lookup table is also generated during this
point. This table assures a speedup inside the main interrupt allowing the software to generate
waveforms up to 400Hz.
\subsubsection{Stage 2: Main timer interrupts}
The main timer interrupt Uses all parameter functions and sets a few parameters that change
depending on the look-up table. These parameters have to be limited otherwise, the interrupt
will take too much time and limit the frequency to a frequency lower than 400Hz.
\subsubsection{Stage 3: PWM timer interrupts}
The PWM Timer interrupts are only lightly used. In normal mode, no interrupt service routine is used, instead, an output pin is toggled when an interrupt is triggered.

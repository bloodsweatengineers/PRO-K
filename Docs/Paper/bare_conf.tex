

\documentclass[conference]{IEEEtran}

\ifCLASSINFOpdf

\else
 
\fi

\hyphenation{op-tical net-works semi-conduc-tor}


\begin{document}

\title{Pulse Width Modulation Controller Software}


\author{\IEEEauthorblockN{Kerim Kilic}
\IEEEauthorblockA{De Haagse Hogeschool}
\and
\IEEEauthorblockN{Dennis van den Berg}
\IEEEauthorblockA{De Haagse Hogeschool}
\and
\IEEEauthorblockN{Folkert Kevelam}
\IEEEauthorblockA{De Haagse Hogeschool}}


\maketitle


\begin{abstract}
We present the high level design of the software which is being used to generate sinusoidal waves from the Universal Four Leg. The Universal Four Leg is a sophisticated piece of hardware which can be configured as a inverter. Using this software, the different parameters of the sinusoidal wave can be manipulated.
\end{abstract}


\IEEEpeerreviewmaketitle



\section{Introduction}
DC voltage have shown great promise in transfering power with less losses than AC voltage. AC voltage is being used in a lot of home appliances. In order to use the voltage of a DC grid for AC applications the Universal Four Leg inverter is being used. With the Universal Four Leg and the software which has been developed in the past months, it is now possible to power various AC Loads from DC grids. With the software which is being described in this paper, the various parameters of the AC output voltages can be manipulated, thus allowing to drive various loads. 

\hfill Delft
 
\hfill December 27, 2019

\section{Communication protocol}
The communication protocol for both the User Interface as for the terminal version of the software is made up of different commands. Some commands require a channel, where the channel is the specific leg the user wants to manipulate it's parameters. The microcontroller always sends an acknowledgement message to the computer, which indicates if the command is accepted or rejected. If the command is accepted the microcontroller will send 'OK' as the acknowledgement message. If the command is rejected the microcontroller will send 'REJ' as the acknowledgement message.

\subsection{User Interface}\label{UI_communication}
The communication between the microcontroller and user interface are made up of messages of 6 bytes each. The first byte represents the start byte which in our case is 0x24, which is also in ASCII equal to the Dollar sign (\$). The second byte represents the command that is being sent over (e.g. the frequency). The third, fourth and fifth byte represent the value that is being sent over the USART communication. The last byte represents the CRC checksum. If the value is N/A, for instance in the case of the command start, then the bytes representing the value will have the value of 0xFF FF FF.

\subsection{Terminal}
In order for the software to also work with terminals like, the Arduino serial monitor or the serial plotter, the terminal communication uses string messages. The syntax for this version of the protocol is:
\\ \{Command\}\_\{Channel\} \{Value\}\textbackslash r \textbackslash n. Where between the brackets come the string representation of the command, channel and value (e.g. frequency\_1 10).

\section{Algorithms}
Hier komt iets over het algoritme

\section{Embedded software architecture}
A large part of the project pertains the control and generation of a PWM waveform described by
user input. To that end, the embedded software must be able to perform its generation duties whilst
recieving or outputting paramters/data. There are multiple ways of achieving said multitasking such as
polling or interrupt-driven execution. Since the required hardware does not have the ability to prioritise
specific interrupts, a pure interrup-driven approach was ruled out, instead, it was chosen to focus on a dual apporach. The routines with a deadline such as the waveform generation would work with interrupt while the routines without a deadline such as user interaction would work with a polling system.

    \subsection{Polling state}
    
    \subsection{Input and output formatting}
    \\r\\n

    \subsection{Waveform generation}

%\section{Embedded software architecture}

%\subsection{Statemachine}
%The statemachine is created so that the user input is processed correctly. The user should always give a start %command first before anything can happen. Then, the user can immediately adjust one parameter or the user does a %prepare. When using prepare, the user can adjust multiple parameters and the command execute loads all these %parameters from the config.

%When execute, then the user will always end up in the start\_state. The execute\_state is no real state (just as %config), because there is continuing with the next state.

%Finally, from each state there is the opportunity to attend the stop\_state, if there is a stop command. It is %also possible to call functions from start\_state and prepare\_state. These functions are not shown in the %statemachine, but there are in the code. These functions are: enable, VFD, ping, info and gather. In the %functions there is briefly something done, and then go back to the original state. This can be the start\_state %or the prepare\_state.

%\subsection{Waveform}
%Iets over de waveform

%\subsection{Parser}
%Iets over de parser

%(vertaling staat hier onder en er moet wat bij komen) Bij de parser is het belangrijk dat er als eerst gekeken %wordt of het commando vanuit de terminal komt of vanuit de GUI. Doordat hiernaar gekeken wordt wordt de juiste %manier gekozen en zo kan er bepaald worden of er een binary of een string command is.

%It is important to have as first look at the command. Because, it can come from the terminal or from the GUI. %is important to check that and than can it be determined if the commands is a binary or string command.

\section{Computer software archictecture}

The architecture of the Python software can be separated in three parts. The first part covers the graphical aspect of the user interface. The second part covers the communication over USART to the microcontroller. The last part acts as the main for the python software.

\subsection{Graphics}
For the graphical part the Python package Tkinter is being used in multiple files. The main file for the graphical part is the file user\_interface.py. This file also acts as the main frame for the User Interface. This file contains the GUI class, which essentially is the main window for our graphical user interface. From this GUI class different objects from different classes from different files are being created in order to form the User Interface as it is. The leg\_data\_tab.py file has a class which creates the different tabs in the UI to show the different settings of the individual legs. The enable\_leg.py file contains a class which is responsible for creating checkbuttons to enable different legs of the Universal Four Leg. The field file contains a class and two subclasses. The first subclass is responsible for creating entry points for data, for instance to set the frequency or amplitude. The second subclass is responsible for creating data displays, where the current settings of the Universal 4 Legs are being displayed. To make sure that the user puts in correct characters and correct values, there is a error\_handler.py file. This file contains a class which checks if the individual entries contain prohibited characters. All of these files and classes combined forms the Graphical User Interface.

\subsection{Communication}
The communication part of the Python software is separated into two files. The command file contains a class which encodes the specific command and value given in the user\_interface.py file into a binary representation as explained in section \ref{UI_communication} in this paper. This binary representation can then be send to the class in the uart\_communication.py file. The class in this file sets up the communication between the microcontroller and computer. It gives an error message if there is no microcontroller present. It has functions to send commands over USART to the microcontroller.

\subsection{Main}
The main code of the computer software is found in the file main.py. In this file an object of the class GUI is created. This file also loops and refreshes the user interface.

\section{Conclusion}
The developed software allows to use both the terminal and a specifically designed User Interface to send commands to the microcontroller.

Het project heeft twee mogelijke manieren om de commando's te verzenden. De ene manier is via de gemaakte UI en de andere manier is via de terminal. De terminal stuurt commando's via binairy en de UI stuurt via string. Via deze twee manieren worden er commando's gestuurd naar de Arduino en deze stuurt de 4 legs aan. Wanneer er een commando's verzonden wordt komt er altijd een acknoedgement die een OK kan bevatten of een REJ.  

Binnen dit project is er gewerkt aan de afhandeling van de commando's. Ook is er aandacht besteed aan de waveform van het signaal. De commando's die doorkomen manipuleren deze waveform

Verder is er gewerkt aan de UI. Deze Python code is opgedeeld in drie delen. graphics, communication en main. De graphics is Tkinter gebruikt. De graphics gebruikt een class met verschillende functies. De GUI heeft een plek waar waardes ingevuld kunnen worden en een plek waar de huidige parameters worden weergegeven. Verder wordt er bij de communicatie gebruikt gemaakt van de binary representation. Wanneer er geen communicatie mogelijk is verschijnt er een foutmelding. De main loops and refreshes the user interface. 

Tot slot is de werkende code getest met een simpel low pass filter. Hiermee is gekeken of er een sinus uit kwam. Vervolgens is er gekeken wat de sinus weergaf wanneer er parameters werden aangepast. In eerste instantie is er met de terminal getest. Later is er ook met de GUI getest.



\section*{Acknowledgment}
The authors gratefully acknowledge the contributions of W. Muhammad, F. Theinert, D. Zuidervliet and P. van Duijssen to the development and support in development of the software.






\begin{thebibliography}{1}

\bibitem{IEEEhowto:kopka}
H.~Kopka and P.~W. Daly, \emph{A Guide to \LaTeX}, 3rd~ed.\hskip 1em plus
  0.5em minus 0.4em\relax Harlow, England: Addison-Wesley, 1999.

\end{thebibliography}

\section{Authors bios and photographs}

[foto]\textbf{Kerim Kilic} [short biography]

[foto]\textbf{Dennis van den Berg} [short biography]

[foto]\textbf{Folkert Kevelam}[short biography]



\end{document}


